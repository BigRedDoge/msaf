% -----------------------------------------------
% Template for ISMIR 2014
% (based on earlier ISMIR templates)
% -----------------------------------------------

\documentclass{article}
\usepackage{ismir2014,amsmath,cite}
\usepackage{graphicx}
\usepackage{amsfonts}

% Title.
% ------
\title{Analysis of Music Segmentation Boundaries:\\ Machines vs Humans}

% Single address
% To use with only one author or several with the same address
% ---------------
%\oneauthor
% {Names should be omitted for double-blind reviewing}
% {Affiliations should be omitted for double-blind reviewing}

% Two addresses
% --------------
%\twoauthors
  %{First author} {School \\ Department}
  %{Second author} {Company \\ Address}

% Three addresses
% --------------
\threeauthors
  {First author} {Affiliation1 \\ {\tt author1@ismir.edu}}
  {Second author} {\bf Retain these fake authors in\\\bf submission to preserve the formatting}
  {Third author} {Affiliation3 \\ {\tt author3@ismir.edu}}

% Four addresses
% --------------
%\fourauthors
%  {First author} {Affiliation1 \\ {\tt author1@ismir.edu}}
%  {Second author}{Affiliation2 \\ {\tt author2@ismir.edu}}
%  {Third author} {Affiliation3 \\ {\tt author3@ismir.edu}}
%  {Fourth author} {Affiliation4 \\ {\tt author4@ismir.edu}}

\begin{document}
%
\maketitle
%
\begin{abstract}
  The identification of segment boundaries in music is a relevant task in Music Information Retrieval
  
\end{abstract}
%
\section{Introduction}\label{sec:introduction}

Music segmentation intro: focusing on boundaries.

Subjectivity of the task.

Analysis of how machines do vs humans. Aim to improve automatic algorithms.

Organization of the paper.

\section{Boundaries Evaluation Metrics}

F-measure at 3 seconds and 0.5. Trim/No trim. Information Gain. Entropy scores.

\section{Machines Identifying Boundaries}\label{sec:eval_desc}

Different approaches: novelty, homogeneity, repetition.

Choosing the five algorithms: OLDA\cite{McFee2014}, Serr\`a\cite{Serra2013},
Foote\cite{Foote1999}, Levy\cite{Levy2008}, SI-PLCA\cite{Weiss2011}.

Talk about the Mean Mutual Agreement and Mean Performance Ground-truth\cite{Holzapfel2012}.

\subsection{Music Segmentation Framework}

Open source project\footnote{Not displayed for revision process.}

Machines.

\subsection{Dataset}

Isophonics, SALAMI\cite{Smith2011}, Cerulean, Ephiphyte.

\subsection{Selection of Hardest and Easiest Tracks}

Plots of MGP and MMA.

\section{Humans Identifying Boundaries}\label{sec:using_method}

Experiment. 5 subjects. 50 tracks: 45 hard, 5 easy.

\subsection{Experiment Setup}

Talk about the SALAMI guidelines.

\section{Machines vs Humans}

Plot comparing MGP-machines vs MGP-humans.

Analyze the answers, display most common problems in a table.

%\begin{table}
 %\begin{center}
 %\begin{tabular}{|l|l|}
  %\hline
  %String value & Numeric value \\
  %\hline
  %Hello ISMIR  & 2014 \\
  %\hline
 %\end{tabular}
%\end{center}
 %\caption{Table captions should be placed below the table.}
 %\label{tab:example}
%\end{table}

%\begin{figure}
 %\centerline{\framebox{
 %\includegraphics[width=\columnwidth]{figure.png}}}
 %\caption{Figure captions should be placed below the figure.}
 %\label{fig:example}
%\end{figure}

\section{Conclusions}

Identified some audio properties that could improve the automatic segmentation algorithms.


%\begin{thebibliography}{citations}

%\bibitem {Author:00}
%E. Author:
%``The Title of the Conference Paper,''
%{\it Proceedings of the International Symposium
%on Music Information Retrieval}, pp.~000--111, 2000.

%\bibitem{Someone:10}
%A. Someone, B. Someone, and C. Someone:
%``The Title of the Journal Paper,''
%{\it Journal of New Music Research},
%Vol.~A, No.~B, pp.~111--222, 2010.

%\bibitem{Someone:04} X. Someone and Y. Someone: {\it Title of the Book},
    %Editorial Acme, Porto, 2012.

%\end{thebibliography}

\bibliography{references}

\end{document}
